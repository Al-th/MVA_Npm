\section{Conclusion et discussions}

Le problème du recalage de nuages de points 3D (mise en référentiel commun des deux nuages) est un problème ouvert depuis un certain nombre d'années. Récemment il est devenu d'autant plus important que les technologies ont démocratisé l'utilisation de données 3D. Dès 1992, Besl et McKay (\cite{bib_icp}) ont proposé un algorithme de recalage dénommé "Iterative Closest Point". Depuis, de nombreuses variantes ont été développées. C'est dans ce cadre que nous avons étudié le modèle probabiliste dénommé Generalized-ICP (GICP) introduit par Segal et al. (\cite{bib_gicp}).\\

Après avoir implémenté à la fois le modèle probabiliste GICP de Segal et al. et l'ICP standard dans sa version analytique, nous avons mis en place un certain nombre de tests et d'expériences afin de comparer les différents algorithmes. Nous avons notamment étudié l'influence de la distance maximale d'appariement et de la transformation initiale sur des données synthétiques (base de donnée de Stanford \cite{Stanford}) et sur des données réelles (base de donnée d'Hannovre \cite{Hannovr}). Enfin nous avons comparé les performances obtenues entre l'ICP standard dans le framework GICP et l'ICP standard dans sa version analytique.\\

Comme présenté dans \cite{bib_gicp}, la variante Plane-to-plane est moins dépendantes de la distance maximale d'appariements que l'ICP standard. En particulier, un seuil trop élevé baisse la précision du recalage de l'ICP standard. Cet effet est moins important dans le cas de la variante Plane-to-plane. D'autre part, nos expériences ont tendances à confirmer un meilleur recalage pour GICP. Nous avons cependant observé que la transformation initiale a l'air de jouer un rôle plus important dans le cas de GICP pour lequel l'algorithme a arrêté de converger dans des conditions initiales trop éloignées de la transformation optimale $\mathbf{T^{*}}$. Enfin nous avons pu vérifier que l'algorithme d'ICP standard développé dans le framework GICP obtenait des résultats identiques à la variante analytique au détriment d'un temps de calcul beaucoup plus important. \\

En conclusion, le framework décrit par Segal et al. généralise de manière formelle les algorithmes d'ICP Point-to-point et Point-to-plane tout ayant permis le développement d'une nouvelle variante (Plane-to-plane). Si ce qui a été avancé dans l'article semble vérifié, il serait intéressant d'optimiser notre implémentation afin d'étudier des nuages de points plus conséquents. D'autre part, nous pourrions créer des jeux de données synthétiques nous permettant d'étudier plus en détail des problématiques spécifiques (occlusions, outliers, recouvrement partiels, échantillonnages différents).
