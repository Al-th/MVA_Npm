\section{Résultats}
\label{chap_results}

Afin d'analyser les performances de la méthode proposée par \cite{bib_gicp}, nous avons tout d'abord chercher à étudier l'influence du critère $d_{max}$ appliquer lors de l'appariement des points, la robustesse de l'algorithme face à des changement de poses différents et la différence de résultats entre le Standard ICP obtenu avec la formule analytique et le Standard ICP obtenu avec le modèle probabilistique présenté dans l'article. (en posant $C_{i}^B=Id$ et $C_{i}^A=0$).\\

Etant donné que GICP utilise l'information des surfaces de chacun des nuages de points, l'algorithme présente donc théoriquement une meilleure robustesse à la différence d'échantillonnage entre les deux nuages de points. Ainsi, l'ensemble des tests seront réalisés à la fois sur une paire de scans synthétiques présentée figure \ref{fig_nuage_bunny} et une paire de scans réelles acquise par l'université de Hannovre présentée \ref{fig_nuage_recalage} . On dispose pour les scans réels de l'estimation de leur poses, i.e une matrice de transformation $T^*$, afin de s'en servir de vérité terrain. Enfin, pour générer la paire de scans synthétiques, une transformation connue $T^{*}$ est appliquée sur le nuage de points synthétiques puis la position des points du nuage transformé est modifié par un bruit gaussien.\\

La transformation initiale $T_0$ est définie comme la transformation exacte $T^*$ auquel on ajoute une erreur entre $\pm1.5m$ et $\pm15^{\circ}$ selon tous les axes. Afin d'évaluer les performances, on mesure l'erreur de position moyenne $||B - \mathbf{T}(A)||_2$ (expliquer comme on gère lorsque les nuages ne sont pas échantillonnés pareil dans la prochaine partie (erreur plus élevée, NN plus compliqué). 
\subsubsection{Etude de l'influence de $d_{max}$}
%Explication de d_max
La mise en correspondance des points des deux nuages est une étape cruciale de l'algorithme ICP. La solution la plus répandue consiste à apparier les points par recherche du plus proche voisin. Cependant pour gérer les problèmes de nuage qui ne se recoupe pas complètement, on retire les points appariés dont la distance est supérieure à une certaine valeur $d_{max}$. En général, le choix de $d_{max}$ est un compromis entre la précision de l'algorithme et sa portée de convergence. Une valeur faible de $d_{max}$
%Etude théorique : +dmax petit, +précis mais moins de chance de convergence et inversement
%Présentation des résultats
%Analyse des résultats
\subsubsection{Etude de l'influence de l'offset}
\subsubsection{Standard ICP par solution analytique}